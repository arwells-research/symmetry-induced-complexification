\section{One-Parameter Distinguishability Symmetry}

We now introduce the additional structural assumption that enables symmetry-induced complexification of the representational Hilbert space.

\begin{definition}[One-parameter distinguishability symmetry]
A one-parameter distinguishability symmetry is a strongly continuous one-parameter family of bounded operators
\[
T : \mathbb{R} \to \mathcal{B}(\overline{\mathcal{V}})
\]
such that for all \(t,s \in \mathbb{R}\) and all \(\psi,\phi \in \overline{\mathcal{V}}\):

\begin{enumerate}

\item \emph{Group property:}
\[
T(t+s) = T(t)T(s),
\qquad
T(0) = I.
\]

\item \emph{Strong continuity:}
\[
\lim_{t \to 0} \|T(t)\psi - \psi\| = 0.
\]

\item \emph{Distinguishability preservation:}
\[
\langle T(t)\psi, T(t)\phi\rangle
=
\langle \psi, \phi\rangle.
\]

\end{enumerate}

\end{definition}

Thus \(T(t)\) is a strongly continuous one-parameter group of orthogonal operators on the real Hilbert space \(\overline{\mathcal{V}}\).

\medskip

\noindent
\emph{Terminological remark.}
Earlier work referred to this symmetry as ``cyclic,'' emphasizing its geometric interpretation as a continuous rotation in representation space. The present terminology emphasizes its operator-theoretic formulation as a strongly continuous one-parameter orthogonal group, which enables precise analysis using Stone's theorem, spectral theory, and functional calculus.

\subsection{Generator of the symmetry}

Since \(T(t)\) is a strongly continuous one-parameter group of orthogonal
operators on the real Hilbert space \(\overline{\mathcal{V}}\), there exists a
unique densely defined skew-adjoint operator
\[
A : \mathcal{D}(A)\subset\overline{\mathcal{V}} \to \overline{\mathcal{V}}
\]
such that
\[
T(t)=e^{tA}
\quad \forall t\in\mathbb{R}.
\]

This result follows from Stone's theorem via complexification of the
representation. Complexifying \(\overline{\mathcal{V}}\) yields a complex
Hilbert space \(\overline{\mathcal{V}}_\mathbb{C}\) on which \(T(t)\) extends
uniquely to a strongly continuous one-parameter unitary group. By Stone's theorem for unitary groups \cite[Theorem VIII.8]{ReedSimonI}; see also
\cite[Section 10.2]{HallQM}, this unitary group admits a unique self-adjoint
generator \(H\). The generator of the original real orthogonal group is then
obtained by restriction of \(iH\) to the invariant real subspace. The original real Hilbert space \(\overline{\mathcal{V}}\)
is invariant under the unitary group, and the restriction of \(iH\) to
\(\overline{\mathcal{V}}\) defines the skew-adjoint generator
\[
A = iH\big|_{\overline{\mathcal{V}}}
\]
of the orthogonal symmetry group. This restriction is well-defined and yields
the unique skew-adjoint generator of the real orthogonal group, by the standard
correspondence between strongly continuous orthogonal groups on real Hilbert
space and skew-adjoint generators obtained via complexification and restriction.

The operator \(A\) is characterized by the strong limit
\[
A\psi
=
\lim_{t\to 0} \frac{T(t)\psi - \psi}{t},
\quad \psi \in \mathcal{D}(A),
\]
with domain consisting of all vectors for which this limit exists.

Skew-adjointness means
\[
\langle A\psi,\phi\rangle
=
-
\langle \psi,A\phi\rangle
\quad \forall \psi,\phi \in \mathcal{D}(A),
\]
and implies that \(A\) is closed and generates the symmetry uniquely
\cite[Theorem VIII.8]{ReedSimonI}.

Since \(A\) is skew-adjoint, the operator
\[
-A^2
\]
is self-adjoint and positive semidefinite. This follows from standard
properties of skew-adjoint operators and the functional calculus for
self-adjoint operators \cite{ReedSimonI,ConwayFA,KadisonRingrose}.

\subsection{Invariant decomposition}

We now define the canonical invariant subspaces associated with the generator.

Define
\[
\mathcal{K}_0 := \ker(A),
\qquad
\mathcal{K}_1 := \overline{\mathrm{ran}(A)}.
\]

The closure in the definition of \(\mathcal{K}_1\) is essential because the range of an unbounded closed operator need not be closed in general.

A standard identity for densely defined closed operators on Hilbert space states that
\[
\overline{\mathrm{ran}(A)} = \ker(A^*)^\perp
\]
(see, e.g., \cite{ReedSimonI,ConwayFA}).

Since \(A\) is skew-adjoint, \(A^* = -A\), and therefore
\[
\ker(A^*) = \ker(A).
\]

Thus
\[
\mathcal{K}_1 = \ker(A)^\perp.
\]

It follows immediately that

\[
\overline{\mathcal{V}} = \mathcal{K}_0 \oplus \mathcal{K}_1
\]

is an orthogonal decomposition into closed subspaces.

Because both subspaces are closed, each inherits a Hilbert space structure from \(\overline{\mathcal{V}}\).

\subsection{Invariance under the symmetry}

The subspaces \(\mathcal{K}_0\) and \(\mathcal{K}_1\) are invariant under the symmetry group.

For \(\psi \in \ker(A)\), one has
\[
T(t)\psi = e^{tA}\psi = \psi,
\]
so the symmetry acts trivially on \(\mathcal{K}_0\).

Since \(T(t)\) is unitary and preserves orthogonality, it follows that
\[
T(t)\mathcal{K}_1 \subset \mathcal{K}_1.
\]

Thus both subspaces are invariant under the symmetry group.

\subsection{Spectral characterization}

Because \(-A^2\) is self-adjoint and positive semidefinite, the spectral theorem provides a projection-valued measure
\[
E(\lambda)
\]
such that
\[
-A^2 = \int_0^\infty \lambda \, dE(\lambda)
\]
(see, e.g., \cite{ReedSimonI,ConwayFA,KadisonRingrose,Dixmier}).

The subspaces defined above admit the spectral characterization

\[
\mathcal{K}_0 = E(\{0\})\,\overline{\mathcal{V}},
\qquad
\mathcal{K}_1 = E((0,\infty))\,\overline{\mathcal{V}}.
\]

Thus \(\mathcal{K}_1\) is precisely the strictly positive spectral subspace of the generator.

\subsection{Structural role of the invariant subspace}

On \(\mathcal{K}_0\), the symmetry acts trivially and induces no additional structure.

On \(\mathcal{K}_1\), the generator acts nontrivially and admits functional calculus constructions that define additional representational structure.

In the following section, we use the functional calculus of the generator restricted to \(\mathcal{K}_1\) to construct the canonical complex structure operator determined by one-parameter distinguishability symmetry, given the real Hilbert representation and strongly continuous symmetry assumption.