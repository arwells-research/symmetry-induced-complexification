\section[Quaternionic Non-Determinacy]{Non-Determinacy of Quaternionic Hilbert Structure under One-Parameter Symmetry}

The preceding sections established that one-parameter distinguishability symmetry induces a canonical complex Hilbert structure on the invariant subspace
\[
\mathcal{K}_1 = \overline{\mathrm{ran}(A)}.
\]
We now clarify the extent to which richer scalar-field structures can be determined by the symmetry.

\subsection{Quaternionic Hilbert structure}

A quaternionic Hilbert space may be characterized, in real Hilbert space terms, by the existence of two bounded operators
\[
J_1, J_2 : \mathcal{K}_1 \to \mathcal{K}_1
\]
satisfying:
\begin{enumerate}
\item Complex structure conditions:
\[
J_1^2 = J_2^2 = -I,
\]
\item Anti-commutation:
\[
J_1 J_2 = - J_2 J_1,
\]
\item Inner product preservation: for each \(k \in \{1,2\}\),
\[
\langle J_k \psi, J_k \phi \rangle
=
\langle \psi, \phi \rangle.
\]
\end{enumerate}
These relations generate the quaternion algebra and induce quaternionic scalar multiplication structure \cite{AdlerQuaternionicQM,VaradarajanGeometryQM}.

\subsection{Symmetry compatibility constraint}

A symmetry-compatible operator must commute with the generator in the
domain-qualified sense appropriate for unbounded operators:
\[
J(\mathcal{D}(A)) \subset \mathcal{D}(A),
\qquad
JA\psi = AJ\psi
\quad \forall \psi \in \mathcal{D}(A).
\]

Equivalently, symmetry-compatible complex structure operators lie in the
commutant of the generator representation in the sense of unbounded operator
theory.

Since \(-A^2\) is self-adjoint, the spectral theorem provides a decomposition of \(\mathcal{K}_1\) into invariant spectral sectors \cite{ReedSimonI,KadisonRingrose,Dixmier}. Each irreducible spectral sector corresponds to a minimal invariant subspace under the symmetry action.

\subsection{Commutant structure on irreducible sectors}

On each irreducible spectral sector, the symmetry acts as planar rotation with fixed frequency. On such a sector, the commutant of the symmetry action is commutative \cite{KadisonRingrose,HallQM}.

\subsection{Sectorwise obstruction}

\begin{theorem}[Sectorwise quaternionic obstruction]
On each irreducible symmetry-invariant spectral sector, no pair of anti-commuting symmetry-compatible complex structure operators can exist.
\end{theorem}

\begin{proof}
Let \(\mathcal{S}\) be an irreducible symmetry-invariant spectral sector. Any symmetry-compatible operator must commute with the symmetry generator, hence lie in the commutant of the symmetry action on \(\mathcal{S}\). On \(\mathcal{S}\) this commutant is commutative, so any two symmetry-compatible operators \(J_1,J_2\) must satisfy
\[
J_1 J_2 = J_2 J_1.
\]
This contradicts the quaternionic anti-commutation requirement \(J_1 J_2 = - J_2 J_1\). Therefore no such pair exists on \(\mathcal{S}\).
\end{proof}

\subsection{Multiplicity and non-canonical structures}

When spectral multiplicity exceeds one, the commutant of \(A\) includes additional operators acting within multiplicity subspaces \cite{KadisonRingrose,Dixmier}. Quaternionic structures may exist abstractly in such multiplicity spaces, but they are not determined canonically by one-parameter symmetry alone. Additional independent symmetry generators would be required to canonically select quaternionic structure \cite{AdlerQuaternionicQM,VaradarajanGeometryQM}.

\subsection{Structural consequence}

One-parameter distinguishability symmetry canonically induces complex Hilbert structure (via functional calculus) on \(\mathcal{K}_1\). Quaternionic Hilbert structure is not determined by one-parameter symmetry alone.
