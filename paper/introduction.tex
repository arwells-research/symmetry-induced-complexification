\section{Introduction}

Recent work established that distinguishability functionals satisfying
explicit algebraic consistency conditions admit a canonical Hilbert space
representation (see \cite{WellsMotionFirstQK_v1}). In particular, signed
residual nonnegativity and third-order obstruction jointly induce a positive
semidefinite quadratic kernel whose minimal realization defines a real
Hilbert space uniquely determined up to isometric isomorphism. Here “distinguishability functional” denotes the algebraically defined separability measure introduced in \cite{WellsMotionFirstQK_v1}, whose algebraic consistency conditions induce the positive semidefinite kernel realized by the minimal Hilbert representation. This
representation provides a geometric encoding of distinguishability relations
derived entirely from algebraic consistency, without introducing probabilistic
interpretation, measurement postulates, dynamical laws, or spacetime structure.

To fix intuition, signed residual nonnegativity ensures that pairwise
distinguishability differences define a positive quadratic form, while
third-order obstruction prevents reduction of distinguishability relations to
representation in lower-dimensional affine space. Together these conditions enforce Hilbert
representability in a manner analogous to the construction of reproducing
kernel Hilbert spaces from positive semidefinite kernels.

The present work analyzes additional structure induced on such real Hilbert
representations by continuous symmetry. Specifically, we consider the case
where distinguishability relations admit a strongly continuous one-parameter
symmetry represented by an orthogonal group on the Hilbert space. Such
symmetries arise naturally whenever distinguishability relations remain
invariant under continuous transformations, and their generators introduce
additional structural constraints on the representation.

Continuous one-parameter symmetry has profound structural consequences. On the
invariant subspace where the symmetry generator acts nontrivially, functional calculus applied to the generator determines a compatible complex structure operator, given the real Hilbert representation and symmetry assumption. Prior work established the existence of such symmetry-compatible
complex structure operators under these conditions, but did not analyze their
uniqueness, representation-intrinsic character, or spectral classification.
The present paper addresses these questions.

This symmetry-based emergence of complex structure contrasts sharply with the
conventional formulation of quantum theory, where complex scalar structure is
assumed as part of the formalism. In the present framework, complex structure follows canonically from symmetry acting on real Hilbert representations of distinguishability, given the real Hilbert representation and symmetry assumption. This raises fundamental questions about
the uniqueness and intrinsicness of complex structure, and about whether
alternative scalar structures, such as quaternionic Hilbert spaces, may arise
canonically under similar symmetry assumptions.

We establish three principal results.

First, we prove that on each irreducible symmetry-invariant spectral sector,
the symmetry generator determines a compatible complex structure uniquely up
to sign. This sign ambiguity corresponds to complex conjugation and does not
affect the resulting complex Hilbert structure up to isometric isomorphism.
The operator constructed canonically from the generator via polar decomposition
and functional calculus provides a distinguished symmetry-compatible complex
structure.

Second, we prove that this complex structure is representation-intrinsic.
Specifically, the canonical isometric isomorphism relating any two minimal
Hilbert realizations of the same distinguishability kernel transports the
symmetry generator and the induced complex structure consistently. Thus the
complex structure depends only on the distinguishability kernel and symmetry,
and not on any particular realization of the Hilbert representation.

Third, we classify the complexified invariant subspace using the spectral
theorem applied to the symmetry generator. This yields a canonical decomposition
into invariant spectral sectors, each inheriting a symmetry-compatible complex
Hilbert structure. We further establish a structural limitation: a single
one-parameter symmetry generator alone does not canonically determine
quaternionic Hilbert structure. Quaternionic structure requires the existence
of additional independent symmetry generators satisfying anticommuting
relations that generate the quaternion algebra, and such relations cannot be
constructed from a single generator alone.

\paragraph{Relation to existing literature.}
The analytic mechanism at the core of this paper—constructing a compatible complex structure from a skew-adjoint generator via polar decomposition, yielding
\[
J = A(-A^2)^{-1/2},
\]
—is classical in operator theory. It appears in Segal's treatment of canonical commutation relations and symplectic transformations \cite{Segal1963}, in the analysis of Bogoliubov transformations in quantum field theory, and in the broader study of real and complex polarizations in geometric quantization \cite{Kostant1970}. The present work does not claim novelty of this analytic construction. Its contribution lies instead in three specific structural results within a representational program based on minimal real Hilbert realizations of distinguishability kernels: sectorwise uniqueness of the induced complex structure up to sign on irreducible symmetry-invariant spectral sectors, representation-intrinsicness under canonical isometric identification of minimal realizations, and a precise obstruction to quaternionic Hilbert structure from a single one-parameter symmetry generator. These results concern the canonical status and uniqueness properties of the construction within the distinguishability representational framework, rather than the analytic construction itself.

All results in this paper are conditional: given a minimal real Hilbert representation of the distinguishability kernel and given a strongly continuous one-parameter orthogonal symmetry, the induced complex structure follows canonically on the active invariant subspace $\mathcal{K}_1$. No claim is made that complex Hilbert structure emerges independently of these assumptions.