\documentclass[11pt]{article}

% --------------------------------------------------
% Core mathematics packages
% --------------------------------------------------

\usepackage{amsmath}
\usepackage{amssymb}
\usepackage{amsthm}
\usepackage{mathrsfs}
\usepackage{bm}

% --------------------------------------------------
% Formatting and hyperlinks
% --------------------------------------------------

\usepackage{geometry}
\usepackage[hidelinks]{hyperref}

\geometry{margin=1in}

% --------------------------------------------------
% Theorem environments
% --------------------------------------------------

\theoremstyle{plain}
\newtheorem{theorem}{Theorem}[section]
\newtheorem{proposition}[theorem]{Proposition}

\theoremstyle{definition}
\newtheorem{definition}[theorem]{Definition}

\theoremstyle{remark}
\newtheorem{remark}[theorem]{Remark}

% --------------------------------------------------
% Title and author
% --------------------------------------------------

\title{Symmetry-Induced Complexification of Distinguishability Representation}
\author{A. R. Wells \\Dual-Frame Research Group}
\date{\today}

% --------------------------------------------------
% Document
% --------------------------------------------------

\begin{document}

\maketitle

\begin{abstract}

Distinguishability functionals satisfying explicit algebraic consistency conditions admit a canonical real Hilbert-space representation uniquely determined up to isometric isomorphism \cite{WellsMotionFirstQK_v1}. Given this real Hilbert representation and a strongly continuous one-parameter distinguishability symmetry, the representation admits a compatible complex Hilbert structure on the invariant subspace where the symmetry generator acts nontrivially.

The present work analyzes the complex structure determined by such symmetry and establishes its uniqueness and representational intrinsicness. We prove that on each irreducible symmetry-invariant spectral sector, the symmetry generator determines a compatible complex structure uniquely up to sign, and that the complex structure constructed via functional calculus of the generator is intrinsic to the distinguishability kernel representation under canonical isometric identification of minimal realizations. Using the spectral theorem, we classify the invariant subspace into symmetry sectors, each inheriting a symmetry-compatible complex Hilbert structure.

We further establish a structural limitation: a single strongly continuous one-parameter symmetry does not canonically determine quaternionic Hilbert structure. Such structure requires additional independent symmetry generators satisfying anticommuting relations generating the quaternion algebra.

The analysis is entirely representational and introduces no probabilistic, measurement-theoretic, dynamical, or spacetime assumptions. These results identify the representational consequences of one-parameter symmetry within the distinguishability kernel framework and the precise conditions under which complex Hilbert structure follows canonically from the real Hilbert representation and symmetry assumption.

\end{abstract}

% --------------------------------------------------
% Paper content
% --------------------------------------------------

\section{Introduction}

Recent work established that distinguishability functionals satisfying
explicit algebraic consistency conditions admit a canonical Hilbert space
representation (see \cite{WellsMotionFirstQK_v1}). In particular, signed
residual nonnegativity and third-order obstruction jointly induce a positive
semidefinite quadratic kernel whose minimal realization defines a real
Hilbert space uniquely determined up to isometric isomorphism. Here “distinguishability functional” denotes the algebraically defined separability measure introduced in \cite{WellsMotionFirstQK_v1}, whose algebraic consistency conditions induce the positive semidefinite kernel realized by the minimal Hilbert representation. This
representation provides a geometric encoding of distinguishability relations
derived entirely from algebraic consistency, without introducing probabilistic
interpretation, measurement postulates, dynamical laws, or spacetime structure.

To fix intuition, signed residual nonnegativity ensures that pairwise
distinguishability differences define a positive quadratic form, while
third-order obstruction prevents reduction of distinguishability relations to
representation in lower-dimensional affine space. Together these conditions enforce Hilbert
representability in a manner analogous to the construction of reproducing
kernel Hilbert spaces from positive semidefinite kernels.

The present work analyzes additional structure induced on such real Hilbert
representations by continuous symmetry. Specifically, we consider the case
where distinguishability relations admit a strongly continuous one-parameter
symmetry represented by an orthogonal group on the Hilbert space. Such
symmetries arise naturally whenever distinguishability relations remain
invariant under continuous transformations, and their generators introduce
additional structural constraints on the representation.

Continuous one-parameter symmetry has profound structural consequences. On the
invariant subspace where the symmetry generator acts nontrivially, functional calculus applied to the generator determines a compatible complex structure operator, given the real Hilbert representation and symmetry assumption. Prior work established the existence of such symmetry-compatible
complex structure operators under these conditions, but did not analyze their
uniqueness, representation-intrinsic character, or spectral classification.
The present paper addresses these questions.

This symmetry-based emergence of complex structure contrasts sharply with the
conventional formulation of quantum theory, where complex scalar structure is
assumed as part of the formalism. In the present framework, complex structure follows canonically from symmetry acting on real Hilbert representations of distinguishability, given the real Hilbert representation and symmetry assumption. This raises fundamental questions about
the uniqueness and intrinsicness of complex structure, and about whether
alternative scalar structures, such as quaternionic Hilbert spaces, may arise
canonically under similar symmetry assumptions.

We establish three principal results.

First, we prove that on each irreducible symmetry-invariant spectral sector,
the symmetry generator determines a compatible complex structure uniquely up
to sign. This sign ambiguity corresponds to complex conjugation and does not
affect the resulting complex Hilbert structure up to isometric isomorphism.
The operator constructed canonically from the generator via polar decomposition
and functional calculus provides a distinguished symmetry-compatible complex
structure.

Second, we prove that this complex structure is representation-intrinsic.
Specifically, the canonical isometric isomorphism relating any two minimal
Hilbert realizations of the same distinguishability kernel transports the
symmetry generator and the induced complex structure consistently. Thus the
complex structure depends only on the distinguishability kernel and symmetry,
and not on any particular realization of the Hilbert representation.

Third, we classify the complexified invariant subspace using the spectral
theorem applied to the symmetry generator. This yields a canonical decomposition
into invariant spectral sectors, each inheriting a symmetry-compatible complex
Hilbert structure. We further establish a structural limitation: a single
one-parameter symmetry generator alone does not canonically determine
quaternionic Hilbert structure. Quaternionic structure requires the existence
of additional independent symmetry generators satisfying anticommuting
relations that generate the quaternion algebra, and such relations cannot be
constructed from a single generator alone.

\paragraph{Relation to existing literature.}
The analytic mechanism at the core of this paper—constructing a compatible complex structure from a skew-adjoint generator via polar decomposition, yielding
\[
J = A(-A^2)^{-1/2},
\]
—is classical in operator theory. It appears in Segal's treatment of canonical commutation relations and symplectic transformations \cite{Segal1963}, in the analysis of Bogoliubov transformations in quantum field theory, and in the broader study of real and complex polarizations in geometric quantization \cite{Kostant1970}. The present work does not claim novelty of this analytic construction. Its contribution lies instead in three specific structural results within a representational program based on minimal real Hilbert realizations of distinguishability kernels: sectorwise uniqueness of the induced complex structure up to sign on irreducible symmetry-invariant spectral sectors, representation-intrinsicness under canonical isometric identification of minimal realizations, and a precise obstruction to quaternionic Hilbert structure from a single one-parameter symmetry generator. These results concern the canonical status and uniqueness properties of the construction within the distinguishability representational framework, rather than the analytic construction itself.

All results in this paper are conditional: given a minimal real Hilbert representation of the distinguishability kernel and given a strongly continuous one-parameter orthogonal symmetry, the induced complex structure follows canonically on the active invariant subspace $\mathcal{K}_1$. No claim is made that complex Hilbert structure emerges independently of these assumptions.

\section{One-Parameter Distinguishability Symmetry}

We now introduce the additional structural assumption that enables symmetry-induced complexification of the representational Hilbert space.

\begin{definition}[One-parameter distinguishability symmetry]
A one-parameter distinguishability symmetry is a strongly continuous one-parameter family of bounded operators
\[
T : \mathbb{R} \to \mathcal{B}(\overline{\mathcal{V}})
\]
such that for all \(t,s \in \mathbb{R}\) and all \(\psi,\phi \in \overline{\mathcal{V}}\):

\begin{enumerate}

\item \emph{Group property:}
\[
T(t+s) = T(t)T(s),
\qquad
T(0) = I.
\]

\item \emph{Strong continuity:}
\[
\lim_{t \to 0} \|T(t)\psi - \psi\| = 0.
\]

\item \emph{Distinguishability preservation:}
\[
\langle T(t)\psi, T(t)\phi\rangle
=
\langle \psi, \phi\rangle.
\]

\end{enumerate}

\end{definition}

Thus \(T(t)\) is a strongly continuous one-parameter group of orthogonal operators on the real Hilbert space \(\overline{\mathcal{V}}\).

\medskip

\noindent
\emph{Terminological remark.}
Earlier work referred to this symmetry as ``cyclic,'' emphasizing its geometric interpretation as a continuous rotation in representation space. The present terminology emphasizes its operator-theoretic formulation as a strongly continuous one-parameter orthogonal group, which enables precise analysis using Stone's theorem, spectral theory, and functional calculus.

\subsection{Generator of the symmetry}

Since \(T(t)\) is a strongly continuous one-parameter group of orthogonal
operators on the real Hilbert space \(\overline{\mathcal{V}}\), there exists a
unique densely defined skew-adjoint operator
\[
A : \mathcal{D}(A)\subset\overline{\mathcal{V}} \to \overline{\mathcal{V}}
\]
such that
\[
T(t)=e^{tA}
\quad \forall t\in\mathbb{R}.
\]

This result follows from Stone's theorem via complexification of the
representation. Complexifying \(\overline{\mathcal{V}}\) yields a complex
Hilbert space \(\overline{\mathcal{V}}_\mathbb{C}\) on which \(T(t)\) extends
uniquely to a strongly continuous one-parameter unitary group. By Stone's theorem for unitary groups \cite[Theorem VIII.8]{ReedSimonI}; see also
\cite[Section 10.2]{HallQM}, this unitary group admits a unique self-adjoint
generator \(H\). The generator of the original real orthogonal group is then
obtained by restriction of \(iH\) to the invariant real subspace. The original real Hilbert space \(\overline{\mathcal{V}}\)
is invariant under the unitary group, and the restriction of \(iH\) to
\(\overline{\mathcal{V}}\) defines the skew-adjoint generator
\[
A = iH\big|_{\overline{\mathcal{V}}}
\]
of the orthogonal symmetry group. This restriction is well-defined and yields
the unique skew-adjoint generator of the real orthogonal group, by the standard
correspondence between strongly continuous orthogonal groups on real Hilbert
space and skew-adjoint generators obtained via complexification and restriction.

The operator \(A\) is characterized by the strong limit
\[
A\psi
=
\lim_{t\to 0} \frac{T(t)\psi - \psi}{t},
\quad \psi \in \mathcal{D}(A),
\]
with domain consisting of all vectors for which this limit exists.

Skew-adjointness means
\[
\langle A\psi,\phi\rangle
=
-
\langle \psi,A\phi\rangle
\quad \forall \psi,\phi \in \mathcal{D}(A),
\]
and implies that \(A\) is closed and generates the symmetry uniquely
\cite[Theorem VIII.8]{ReedSimonI}.

Since \(A\) is skew-adjoint, the operator
\[
-A^2
\]
is self-adjoint and positive semidefinite. This follows from standard
properties of skew-adjoint operators and the functional calculus for
self-adjoint operators \cite{ReedSimonI,ConwayFA,KadisonRingrose}.

\subsection{Invariant decomposition}

We now define the canonical invariant subspaces associated with the generator.

Define
\[
\mathcal{K}_0 := \ker(A),
\qquad
\mathcal{K}_1 := \overline{\mathrm{ran}(A)}.
\]

The closure in the definition of \(\mathcal{K}_1\) is essential because the range of an unbounded closed operator need not be closed in general.

A standard identity for densely defined closed operators on Hilbert space states that
\[
\overline{\mathrm{ran}(A)} = \ker(A^*)^\perp
\]
(see, e.g., \cite{ReedSimonI,ConwayFA}).

Since \(A\) is skew-adjoint, \(A^* = -A\), and therefore
\[
\ker(A^*) = \ker(A).
\]

Thus
\[
\mathcal{K}_1 = \ker(A)^\perp.
\]

It follows immediately that

\[
\overline{\mathcal{V}} = \mathcal{K}_0 \oplus \mathcal{K}_1
\]

is an orthogonal decomposition into closed subspaces.

Because both subspaces are closed, each inherits a Hilbert space structure from \(\overline{\mathcal{V}}\).

\subsection{Invariance under the symmetry}

The subspaces \(\mathcal{K}_0\) and \(\mathcal{K}_1\) are invariant under the symmetry group.

For \(\psi \in \ker(A)\), one has
\[
T(t)\psi = e^{tA}\psi = \psi,
\]
so the symmetry acts trivially on \(\mathcal{K}_0\).

Since \(T(t)\) is unitary and preserves orthogonality, it follows that
\[
T(t)\mathcal{K}_1 \subset \mathcal{K}_1.
\]

Thus both subspaces are invariant under the symmetry group.

\subsection{Spectral characterization}

Because \(-A^2\) is self-adjoint and positive semidefinite, the spectral theorem provides a projection-valued measure
\[
E(\lambda)
\]
such that
\[
-A^2 = \int_0^\infty \lambda \, dE(\lambda)
\]
(see, e.g., \cite{ReedSimonI,ConwayFA,KadisonRingrose,Dixmier}).

The subspaces defined above admit the spectral characterization

\[
\mathcal{K}_0 = E(\{0\})\,\overline{\mathcal{V}},
\qquad
\mathcal{K}_1 = E((0,\infty))\,\overline{\mathcal{V}}.
\]

Thus \(\mathcal{K}_1\) is precisely the strictly positive spectral subspace of the generator.

\subsection{Structural role of the invariant subspace}

On \(\mathcal{K}_0\), the symmetry acts trivially and induces no additional structure.

On \(\mathcal{K}_1\), the generator acts nontrivially and admits functional calculus constructions that define additional representational structure.

In the following section, we use the functional calculus of the generator restricted to \(\mathcal{K}_1\) to construct the canonical complex structure operator determined by one-parameter distinguishability symmetry, given the real Hilbert representation and strongly continuous symmetry assumption.

\section{Construction of the Symmetry-Induced Complex Structure}

We now construct the canonical complex structure operator determined by one-parameter distinguishability symmetry and show that it equips the invariant subspace \(\mathcal{K}_1\) with the structure of a complex Hilbert space.

\subsection{Definition of the complex structure operator}

As established in Section 2, the invariant subspace admits the characterization
\[
\mathcal{K}_1 = \ker(A)^\perp = E((0,\infty))\,\overline{\mathcal{V}}.
\]

Since \(A\) is skew-adjoint, the operator
\[
-A^2
\]
is densely defined, self-adjoint, and positive semidefinite
\cite[Theorem VIII.4]{ReedSimonI}.

By the spectral theorem for self-adjoint operators
\cite[Theorem VII.2]{ReedSimonI}, there exists a projection-valued measure
\(E(\lambda)\) such that
\[
-A^2 = \int_0^\infty \lambda \, dE(\lambda).
\]

On the invariant subspace \(\mathcal{K}_1\), the operator \(-A^2\) has trivial kernel. The functional calculus therefore defines the positive self-adjoint operator
\[
|A| := (-A^2)^{1/2}
=
\int_{(0,\infty)} \lambda^{1/2} \, dE(\lambda),
\]
with dense domain equal to \(\mathcal K_1\)
\cite[Theorem VIII.6]{ReedSimonI}.

The inverse operator
\[
|A|^{-1}
=
(-A^2)^{-1/2}
=
\int_{(0,\infty)} \lambda^{-1/2} \, dE(\lambda)
\]
is defined as a bounded operator on \(\mathcal K_1\) when interpreted via the polar decomposition of the normal operator \(A\), where it appears in the bounded product
\[
J = A|A|^{-1}
\]
defining the partial isometry \(J\)
\cite[Theorem VI.13; Theorem VIII.6]{ReedSimonI}.

Since \(A\) is a closed densely defined operator, it admits a polar decomposition
\[
A = J |A|
\]
where \(J\) is a partial isometry with initial space
\[
\overline{\mathrm{ran}(|A|)} = \ker(A)^\perp = \mathcal{K}_1
\]
and final space
\[
\overline{\mathrm{ran}(A)} = \mathcal{K}_1
\]
\cite[Theorem VI.13]{ReedSimonI}.

Because the initial and final spaces coincide, the partial isometry \(J\) restricts to a unitary operator on \(\mathcal{K}_1\).

Multiplying the polar decomposition on the right by \(|A|^{-1}\) yields
\[
J = A |A|^{-1}
=
A(-A^2)^{-1/2}
\quad \text{on } \mathcal{K}_1.
\]

Since \(|A|^{-1}\) is bounded on \(\mathcal{K}_1\), it follows that \(J\) is bounded on \(\mathcal{K}_1\).

This operator is uniquely determined by the symmetry generator through its polar decomposition, given the real Hilbert representation and symmetry assumption.

\subsection{Algebraic properties of the complex structure operator}

We now establish the fundamental structural properties of \(J\).

\begin{proposition}
The operator \(J\) satisfies
\[
J^2 = -I
\]
on the invariant subspace \(\mathcal{K}_1\).
\end{proposition}

\begin{proof}

Since \(A\) is skew-adjoint, it is a normal operator:
\[
A^*A = AA^*.
\]

Therefore, by the polar decomposition theorem for closed normal operators,
the partial isometry \(J\) in the polar decomposition
\[
A = J|A|
\]
commutes with \(|A|\) and all bounded Borel functions of \(|A|\)
\cite[Theorem VI.13; Theorem VIII.6]{ReedSimonI}.

On the invariant subspace \(\mathcal{K}_1 = \ker(A)^\perp\), the operator
\(|A|\) has trivial kernel and therefore admits a bounded inverse.

Multiplying the polar decomposition on the left by \(|A|^{-1}\) gives
\[
|A|^{-1}A
=
|A|^{-1}J|A|.
\]

Using the commutation of \(J\) and \(|A|\), we obtain
\[
|A|^{-1}J|A|
=
J|A|^{-1}|A|
=
J.
\]

Thus
\[
|A|^{-1}A = J.
\]

Multiplying both sides on the right by \(|A|^{-1}A\) yields
\[
J^2
=
|A|^{-1}A|A|^{-1}A
=
|A|^{-2}A^2.
\]

Since
\[
A^2 = -|A|^2,
\]

it follows that
\[
J^2 = -I
\quad \text{on } \mathcal{K}_1.
\]

\end{proof}

\begin{proposition}
The operator \(J\) is skew-adjoint:
\[
J^* = -J.
\]
\end{proposition}

\begin{proof}

Since \(A\) is skew-adjoint and \(|A|^{-1}\) is self-adjoint,
\[
J^*
=
\left(A|A|^{-1}\right)^*
=
|A|^{-1} A^*
=
-|A|^{-1} A.
\]

Using commutation of \(J\) and \(|A|\),
\[
|A|^{-1}A
=
|A|^{-1}J|A|
=
J.
\]

Therefore,
\[
J^* = -J.
\]

\end{proof}

\begin{proposition}
The operator \(J\) is orthogonal:
\[
\langle J\psi, J\phi \rangle
=
\langle \psi, \phi \rangle
\quad \forall \psi,\phi \in \mathcal{K}_1.
\]
\end{proposition}

\begin{proof}

Using skew-adjointness and \(J^2=-I\),
\[
\langle J\psi, J\phi \rangle
=
\langle \psi, J^*J \phi \rangle
=
\langle \psi, (-J^2)\phi \rangle
=
\langle \psi, \phi \rangle.
\]

\end{proof}

\subsection{Induced complex Hilbert structure}

We now use \(J\) to define complex scalar multiplication on \(\mathcal{K}_1\).

For real scalars \(a,b \in \mathbb{R}\), define
\[
(a+ib)\psi := a\psi + bJ\psi.
\]

Since \(J^2=-I\), this defines a valid complex vector space structure.

Define the complex inner product
\[
\langle \psi, \phi \rangle_{\mathbb{C}}
:=
\langle \psi, \phi \rangle
-
i\langle \psi, J\phi \rangle.
\]

\begin{theorem}[Complex structure compatibility]
Given the real Hilbert representation and strongly continuous symmetry assumption, the invariant subspace \(\mathcal{K}_1\), equipped with this scalar multiplication and inner product, is a complex Hilbert space.
\end{theorem}

\begin{proof}

We verify conjugate symmetry. Using the definition of the complex inner product and skew-adjointness of \(J\), we compute

\begin{align*}
\overline{\langle \phi,\psi\rangle_{\mathbb{C}}}
&=
\overline{
\langle \phi,\psi\rangle
-
i\langle \phi,J\psi\rangle
} \\
&=
\langle \psi,\phi\rangle
+
i\langle \phi,J\psi\rangle \\
&=
\langle \psi,\phi\rangle
-
i\langle \psi,J\phi\rangle \\
&=
\langle \psi,\phi\rangle_{\mathbb{C}},
\end{align*}

where the third line uses skew-adjointness of \(J\), namely
\(
\langle \phi,J\psi\rangle = -\langle J\phi,\psi\rangle = -\langle \psi,J\phi\rangle.
\)

Positive definiteness follows from positivity of the original inner product.

The induced norm satisfies
\[
\|\psi\|_{\mathbb{C}}^2
=
\langle \psi,\psi\rangle_{\mathbb{C}}
=
\|\psi\|^2.
\]

Since \(\mathcal{K}_1\) is closed in the complete real Hilbert space \(\overline{\mathcal{V}}\), it is complete in this norm.

\end{proof}

Thus, given the real Hilbert representation and strongly continuous symmetry assumption, one-parameter distinguishability symmetry determines a canonical complex Hilbert structure on the invariant subspace \(\mathcal{K}_1\).


\section[Uniqueness and Canonical Status]{Uniqueness and Canonical Status of the Symmetry-Induced Complex Structure}

In the preceding section, we constructed a complex structure operator
\[
J := A(-A^2)^{-1/2}
\]
on the invariant subspace
\[
\mathcal{K}_1 = \overline{\mathrm{ran}(A)} = \ker(A)^\perp.
\]

We now analyze the extent to which this complex structure is uniquely determined by the symmetry generator.

\subsection{Symmetry-compatible complex structures}

We begin by formalizing compatibility with the symmetry.

\begin{definition}
A symmetry-compatible complex structure on \(\mathcal{K}_1\) is a bounded linear operator
\[
J' : \mathcal{K}_1 \to \mathcal{K}_1
\]
satisfying:

\begin{enumerate}

\item Complex structure condition:
\[
(J')^2 = -I,
\]

\item Inner product preservation:
\[
\langle J'\psi, J'\phi \rangle
=
\langle \psi, \phi \rangle
\quad \forall \psi,\phi \in \mathcal{K}_1,
\]

\item Symmetry compatibility (domain-qualified commutation):
\[
J'(\mathcal{D}(A)) \subset \mathcal{D}(A),
\qquad
J'A\psi = AJ'\psi
\quad \forall \psi \in \mathcal{D}(A).
\]

\end{enumerate}

\end{definition}

This condition is equivalent to commutation with the strongly continuous symmetry group:
\[
J'T(t) = T(t)J'
\quad \forall t \in \mathbb{R},
\]
by Stone's theorem for strongly continuous orthogonal groups
\cite[Theorem VIII.4]{ReedSimonI}.

Thus symmetry-compatible complex structures lie in the commutant of the symmetry representation.

\subsection{Spectral decomposition and multiplicity structure}

Since \(A\) is skew-adjoint, the operator \(-A^2\) is self-adjoint and positive semidefinite. By the spectral theorem, there exists a projection-valued measure \(E(\cdot)\) such that
\[
-A^2 = \int_0^\infty \lambda \, dE(\lambda),
\]
and the Hilbert space decomposes into spectral subspaces
\[
\overline{\mathcal V}
=
E(\{0\})\overline{\mathcal V}
\;\oplus\;
E((0,\infty))\overline{\mathcal V}
=
\mathcal K_0 \oplus \mathcal K_1
\cite[Theorem VII.2]{ReedSimonI}.
\]

In a spectral representation of \(-A^2\), the invariant subspace \(\mathcal{K}_1\) may be realized as a direct integral
\[
\mathcal{K}_1
\cong
\int_{(0,\infty)}^\oplus H_\lambda \, d\mu(\lambda),
\]
where each fiber \(H_\lambda\) is a Hilbert space whose dimension equals the spectral multiplicity at \(\lambda\)
\cite[Section VII.2]{ReedSimonI}.

In this representation, the operator \(-A^2\) acts as multiplication:
\[
((-A^2)\psi)(\lambda) = \lambda \psi(\lambda),
\]
and operators commuting with the symmetry act fiberwise.

Multiplicity-one fibers correspond to irreducible symmetry sectors, while higher multiplicity fibers admit additional commuting operators acting internally within multiplicity spaces.

\subsection{Commutant structure on irreducible sectors}

We now characterize symmetry-compatible operators on irreducible sectors.

\begin{definition}
A symmetry-invariant subspace \(\mathcal S \subset \mathcal K_1\) is irreducible if it contains no nontrivial closed subspace invariant under the symmetry group \(T(t)\).
\end{definition}

On such a subspace, the generator restricts to a skew-adjoint operator with spectrum consisting of a single positive frequency \(\omega>0\). The symmetry acts as planar rotation:
\[
T(t)|_{\mathcal S}
=
\begin{pmatrix}
\cos(\omega t) & -\sin(\omega t) \\
\sin(\omega t) & \cos(\omega t)
\end{pmatrix}
\]
in a suitable orthonormal basis
\cite{HallQM,VaradarajanGeometryQM}.

For real irreducible rotation representations, the commutant algebra consists precisely of operators of the form
\[
aI + bJ,
\quad a,b \in \mathbb R,
\]
where \(J\) is the canonical complex structure operator obtained via functional calculus of the generator.

This follows from the classification of irreducible real representations of one-parameter rotation groups and their commutant algebras
\cite{KadisonRingrose,VaradarajanGeometryQM}.

\subsection{Sectorwise uniqueness}

We now establish uniqueness of symmetry-compatible complex structure on irreducible sectors.

\begin{theorem}[Sectorwise uniqueness up to sign]
Let \(\mathcal S \subset \mathcal K_1\) be an irreducible symmetry-invariant subspace.

Then any symmetry-compatible complex structure operator
\[
J' : \mathcal S \to \mathcal S
\]
satisfies
\[
J' = \pm J
\quad \text{on } \mathcal S.
\]
\end{theorem}

\begin{proof}

Since \(J'\) commutes with the symmetry and \(\mathcal S\) is irreducible, the commutant characterization implies that
\[
J' = aI + bJ
\]
for some real scalars \(a,b\).

The complex structure condition gives
\[
(J')^2 = -I.
\]

Substituting,
\[
(aI + bJ)^2
=
(a^2 - b^2)I + 2abJ.
\]

Equating with \(-I\) yields
\[
2ab = 0,
\qquad
a^2 - b^2 = -1.
\]

If \(b=0\), then \(J' = aI\), which cannot satisfy \((J')^2 = -I\) for real \(a\).

Thus \(a=0\), and
\[
J' = bJ.
\]

The remaining condition gives
\[
b^2 = 1,
\]
and therefore
\[
J' = \pm J.
\]

\end{proof}

\subsection{Multiplicity and global structure}

When spectral multiplicity exceeds one, the commutant algebra includes additional operators acting within multiplicity fibers
\cite{KadisonRingrose,Dixmier}.

Such operators correspond to internal symmetries of multiplicity spaces and are not uniquely determined by the symmetry generator itself.

By contrast, the operator
\[
J = A(-A^2)^{-1/2}
\]
is defined canonically through functional calculus of the generator and requires no additional choices.

It therefore provides a distinguished symmetry-compatible complex structure determined directly by the symmetry representation.

\subsection{Sign ambiguity and canonical equivalence}

The operator \(-J\) also satisfies all defining properties of a symmetry-compatible complex structure.

This corresponds to complex conjugation and does not alter the isomorphism class of the resulting complex Hilbert space.

Thus, on each irreducible symmetry sector, the symmetry determines a unique compatible complex structure up to sign.

Globally, the functional-calculus operator \(J\) provides the canonical symmetry-induced complex structure associated with the one-parameter symmetry generator.


\section{Representation-Intrinsicness of the Complex Structure}

In the preceding sections, we established that, given the real Hilbert representation and strongly continuous symmetry assumption, one-parameter distinguishability symmetry determines a canonical complex structure operator
\[
J := A(-A^2)^{-1/2}
\]
on the invariant subspace \(\mathcal{K}_1\) of the canonical real Hilbert representation.

We now show that this complex structure is intrinsic to the distinguishability kernel representation itself. Specifically, we prove that the canonical isometric isomorphism relating any two minimal Hilbert representations transports both the symmetry generator and the induced complex structure consistently.

\subsection{Canonical isometric isomorphism}

Let
\[
(\overline{\mathcal{V}}_1, \psi_1)
\quad \text{and} \quad
(\overline{\mathcal{V}}_2, \psi_2)
\]
be two minimal Hilbert representations of the same distinguishability kernel.

By the representational uniqueness established in the preceding reconstruction, there exists a unique isometric isomorphism
\[
U : \overline{\mathcal{V}}_1 \to \overline{\mathcal{V}}_2
\]
such that
\[
U \psi_1(\gamma) = \psi_2(\gamma)
\quad \forall \gamma.
\]

This isomorphism preserves the inner product:
\[
\langle U\psi, U\phi \rangle = \langle \psi, \phi \rangle,
\]
and therefore is unitary in the real Hilbert-space sense.

\subsection{Transport of symmetry generators}

Suppose the symmetry is represented in the first representation by a strongly continuous one-parameter orthogonal group
\[
T_1(t) = e^{tA_1},
\]
with skew-adjoint generator \(A_1\).

Define the transported symmetry in the second representation by
\[
T_2(t) := U T_1(t) U^{-1}.
\]

Then \(T_2(t)\) is also a strongly continuous one-parameter orthogonal group.

By Stone's theorem, it admits a unique skew-adjoint generator \(A_2\).

Standard results on transport of generators under unitary equivalence imply that
\[
U(\mathcal{D}(A_1)) = \mathcal{D}(A_2),
\]
and
\[
A_2 U \psi = U A_1 \psi
\quad \forall \psi \in \mathcal{D}(A_1)
\]
\cite[Theorem VIII.8]{ReedSimonI}; see also \cite[Section 10.2]{HallQM}.

Equivalently,
\[
A_2 = U A_1 U^{-1}.
\]

Thus the symmetry generator is transported covariantly under representational isomorphism.

\subsection{Transport of the positive operator and its functional calculus}

Since \(A_1\) and \(A_2\) are skew-adjoint and unitarily equivalent, the associated positive self-adjoint operators
\[
|A_1| := (-A_1^2)^{1/2},
\qquad
|A_2| := (-A_2^2)^{1/2}
\]
satisfy
\[
|A_2| = U |A_1| U^{-1}.
\]

By the spectral theorem, functional calculus of self-adjoint operators is covariant under unitary equivalence. In particular,
\[
|A_2|^{-1}
=
U |A_1|^{-1} U^{-1}
\]
on the corresponding invariant subspaces
\cite{ReedSimonI,KadisonRingrose,Dixmier}.

\subsection{Transport of the polar decomposition}

Since each generator admits a polar decomposition,
\[
A_1 = J_1 |A_1|,
\qquad
A_2 = J_2 |A_2|,
\]
where \(J_1\) and \(J_2\) are the associated partial isometries (unitary on the invariant subspaces),
we may substitute the transported operators to obtain

\[
A_2
=
U A_1 U^{-1}
=
U J_1 |A_1| U^{-1}
=
(U J_1 U^{-1})(U |A_1| U^{-1})
=
(U J_1 U^{-1}) |A_2|.
\]

By uniqueness of the partial isometry in the polar decomposition of a closed operator
\cite[Theorem VI.13]{ReedSimonI},
it follows that

\[
J_2 = U J_1 U^{-1}.
\]

\begin{definition}[Representation-intrinsic structure]
Let a distinguishability kernel and symmetry be given, and let
\[
(\overline{\mathcal{V}}_1, T_1(t))
\quad \text{and} \quad
(\overline{\mathcal{V}}_2, T_2(t))
\]
be two minimal Hilbert space representations related by the canonical
isometric isomorphism
\[
U : \overline{\mathcal{V}}_1 \to \overline{\mathcal{V}}_2.
\]

A structural feature of the representation is called
\emph{representation-intrinsic} if it is preserved under this canonical
isometric isomorphism. That is, the corresponding operators or structures
are transported by \(U\) according to
\[
\mathcal{S}_2 = U \mathcal{S}_1 U^{-1}.
\]

Equivalently, representation-intrinsic structure depends only on the
distinguishability kernel and symmetry representation itself, and not on
any particular Hilbert space realization.
\end{definition}

\subsection{Intrinsicness theorem}

\begin{theorem}[Representation-intrinsicness]
Let
\[
(\overline{\mathcal{V}}_1, J_1)
\quad \text{and} \quad
(\overline{\mathcal{V}}_2, J_2)
\]
be complex Hilbert structures determined by symmetry generators arising from two minimal real Hilbert representations of the same distinguishability kernel and symmetry.

Then the canonical isometric isomorphism
\[
U : \overline{\mathcal{V}}_1 \to \overline{\mathcal{V}}_2
\]
satisfies
\[
U J_1 = J_2 U.
\]

\end{theorem}

\begin{proof}

The generators satisfy
\[
A_2 = U A_1 U^{-1},
\]
and therefore the associated positive operators satisfy
\[
|A_2| = U |A_1| U^{-1}.
\]

The polar decompositions are uniquely determined by these operators. Substituting the transported operators into the polar decomposition and using uniqueness of the partial isometry yields
\[
J_2 = U J_1 U^{-1}.
\]

\end{proof}

\subsection{Structural consequence}

This result shows that, given the real Hilbert representation and symmetry assumption, the symmetry-induced complex structure is intrinsic to the distinguishability kernel together with its symmetry representation.

It does not depend on arbitrary choices in the Hilbert-space realization.

Any two minimal Hilbert representations of the same distinguishability kernel and symmetry yield canonically isomorphic complex Hilbert structures.

Thus, given the real Hilbert representation and strongly continuous symmetry assumption, complex scalar multiplication follows as a representation-intrinsic structural consequence of one-parameter distinguishability symmetry.


\section{Spectral Classification of the Complexified Subspace}

The preceding sections established that, given the real Hilbert representation and strongly continuous symmetry assumption, one-parameter distinguishability symmetry determines a canonical complex Hilbert structure on the invariant subspace
\[
\mathcal{K}_1 := \overline{\mathrm{ran}(A)}.
\]

We now classify this structure using the spectral decomposition of the symmetry generator.

\subsection{Spectral resolution of the generator}

Since \(A\) is skew-adjoint, the operator \(-A^2\) is self-adjoint and positive semidefinite
\cite[Theorem VII.2]{ReedSimonI}.

By the spectral theorem for self-adjoint operators, there exists a projection-valued measure
\[
E(\lambda)
\]
defined on Borel subsets of \([0,\infty)\) such that

\[
-A^2
=
\int_0^\infty \lambda \, dE(\lambda).
\]

This spectral resolution provides a canonical decomposition of the Hilbert space into invariant subspaces corresponding to spectral regions of the symmetry generator.

\subsection{Characterization of the invariant subspace}

As established in Section 2.2, the invariant subspace admits the characterization
\[
\mathcal{K}_1 = \ker(A)^\perp = E((0,\infty))\,\overline{\mathcal{V}}.
\]

Thus \(\mathcal{K}_1\) is precisely the strictly positive spectral subspace of the operator \(-A^2\), corresponding to the portion of the representation on which the symmetry generator acts nontrivially.

\subsection{Spectral sector decomposition}

For any measurable subset \(\Delta \subset (0,\infty)\), define the invariant subspace

\[
\mathcal{K}_\Delta := E(\Delta)\,\overline{\mathcal{V}}.
\]

These spectral subspaces satisfy the standard properties of spectral decompositions:

\begin{enumerate}

\item Orthogonality:
\[
\mathcal{K}_{\Delta_1} \perp \mathcal{K}_{\Delta_2}
\quad \text{if } \Delta_1 \cap \Delta_2 = \varnothing,
\]

\item Invariance under symmetry:
\[
T(t)\mathcal{K}_\Delta \subset \mathcal{K}_\Delta,
\]

\item Invariance under functional calculus:
\[
f(A)\mathcal{K}_\Delta \subset \mathcal{K}_\Delta
\]
for any bounded Borel function \(f\),

\item Completeness:
For any measurable partition of \((0,\infty)\), the invariant subspace
\(\mathcal{K}_1\) decomposes as the closure of the orthogonal sum of the
corresponding spectral subspaces. In the general case, this decomposition
takes the form of a direct integral:
\[
\mathcal{K}_1
=
\int_{(0,\infty)}^\oplus H_\lambda \, d\mu(\lambda),
\]
where each fiber \(H_\lambda\) is invariant under the symmetry and the
operator \(-A^2\) acts fiberwise as multiplication by \(\lambda\)
\cite[Theorem VII.2; Section VII.2]{ReedSimonI}.

\end{enumerate}

In general, this decomposition takes the form of a direct integral of invariant Hilbert spaces indexed by the spectrum
\cite{ReedSimonI,ConwayFA,Dixmier}.

\subsection{Compatibility with the complex structure operator}

The complex structure operator

\[
J := A(-A^2)^{-1/2}
\]

is defined via functional calculus of the self-adjoint operator \(-A^2\).

Functional calculus ensures that \(J\) commutes with all spectral projections:

\[
J E(\Delta) = E(\Delta) J
\quad \forall \Delta \subset (0,\infty)
\]
\cite{ReedSimonI,KadisonRingrose,Dixmier}.

Thus each spectral subspace \(\mathcal{K}_\Delta\) is invariant under \(J\).

Each spectral sector therefore inherits a compatible complex Hilbert space structure.

\subsection{Multiplicity structure}

The spectral theorem represents \(\mathcal{K}_1\) as a direct integral of Hilbert spaces over the spectrum:

\[
\mathcal{K}_1
=
\int^\oplus H_\lambda \, d\mu(\lambda)
\]
\cite{ReedSimonI,ConwayFA,Dixmier}.

Each fiber \(H_\lambda\) corresponds to a spectral component with multiplicity equal to \(\dim(H_\lambda)\), which may be finite or infinite.

On each such spectral component, the symmetry generator acts as a rotation operator with frequency determined by \(\sqrt{\lambda}\).

The functional calculus construction defines the complex structure operator fiberwise across this decomposition.

In the presence of spectral multiplicity greater than one, the symmetry generator alone does not uniquely determine the internal operator structure within each multiplicity fiber. The functional-calculus operator
\[
J = A(-A^2)^{-1/2}
\]
remains canonically defined and provides a symmetry-compatible complex structure on the invariant subspace. However, additional operators acting within multiplicity fibers may exist that also commute with the symmetry generator and therefore are not fixed by the generator alone. In particular, quaternionic Hilbert structure requires the existence of additional complex structure operators satisfying anticommuting relations generating the quaternion algebra. Such additional structure cannot be constructed from a single symmetry generator and would require the presence of further independent symmetry generators imposing additional algebraic constraints.

\paragraph{Sectorwise uniqueness and multiplicity fibers.}
The uniqueness of the complex structure operator established earlier holds sectorwise on irreducible invariant spectral components of the symmetry generator. In the presence of spectral multiplicity, each fiber $H_\lambda$ may admit additional orthogonal degrees of freedom corresponding to multiplicity rotations that commute with the generator. These internal transformations do not alter the functional-calculus construction of
\[
J = A(-A^2)^{-1/2},
\]
which remains canonically determined by the generator itself. However, they imply that uniqueness holds relative to the spectral sector decomposition rather than globally across multiplicity fibers. Further reduction of this residual internal freedom requires the presence of additional independent symmetry generators imposing further structural constraints.

\subsection{Classification theorem}

\begin{theorem}[Spectral classification]
Given the real Hilbert representation and strongly continuous symmetry assumption, the induced complex Hilbert structure decomposes canonically according to the spectral resolution of the symmetry generator.

Each spectral sector inherits a compatible complex Hilbert structure determined by functional calculus of the generator.

The operator
\[
J = A(-A^2)^{-1/2}
\]
defines the canonical symmetry-compatible complex structure on \(\mathcal{K}_1\).
\end{theorem}

\begin{proof}

The spectral theorem provides the invariant decomposition into spectral subspaces.

Functional calculus defines the operator \(J\), which commutes with spectral projections.

Thus each spectral sector is invariant under \(J\) and inherits a compatible complex Hilbert structure.

Since \(J\) is defined uniquely by functional calculus of the generator, the resulting complex structure is canonically determined by the symmetry.

\end{proof}

\subsection{Structural interpretation}

This classification shows that, given the real Hilbert representation and strongly continuous symmetry assumption, complex Hilbert structure follows canonically on the invariant subspace from the action of the symmetry generator.

Each spectral sector corresponds to an independent symmetry mode, and the complex structure is induced canonically by the symmetry generator.

No additional representational assumptions are required beyond the real Hilbert representation and symmetry assumption.


\section[Quaternionic Non-Determinacy]{Non-Determinacy of Quaternionic Hilbert Structure under One-Parameter Symmetry}

The preceding sections established that one-parameter distinguishability symmetry induces a canonical complex Hilbert structure on the invariant subspace
\[
\mathcal{K}_1 = \overline{\mathrm{ran}(A)}.
\]
We now clarify the extent to which richer scalar-field structures can be determined by the symmetry.

\subsection{Quaternionic Hilbert structure}

A quaternionic Hilbert space may be characterized, in real Hilbert space terms, by the existence of two bounded operators
\[
J_1, J_2 : \mathcal{K}_1 \to \mathcal{K}_1
\]
satisfying:
\begin{enumerate}
\item Complex structure conditions:
\[
J_1^2 = J_2^2 = -I,
\]
\item Anti-commutation:
\[
J_1 J_2 = - J_2 J_1,
\]
\item Inner product preservation: for each \(k \in \{1,2\}\),
\[
\langle J_k \psi, J_k \phi \rangle
=
\langle \psi, \phi \rangle.
\]
\end{enumerate}
These relations generate the quaternion algebra and induce quaternionic scalar multiplication structure \cite{AdlerQuaternionicQM,VaradarajanGeometryQM}.

\subsection{Symmetry compatibility constraint}

A symmetry-compatible operator must commute with the generator in the
domain-qualified sense appropriate for unbounded operators:
\[
J(\mathcal{D}(A)) \subset \mathcal{D}(A),
\qquad
JA\psi = AJ\psi
\quad \forall \psi \in \mathcal{D}(A).
\]

Equivalently, symmetry-compatible complex structure operators lie in the
commutant of the generator representation in the sense of unbounded operator
theory.

Since \(-A^2\) is self-adjoint, the spectral theorem provides a decomposition of \(\mathcal{K}_1\) into invariant spectral sectors \cite{ReedSimonI,KadisonRingrose,Dixmier}. Each irreducible spectral sector corresponds to a minimal invariant subspace under the symmetry action.

\subsection{Commutant structure on irreducible sectors}

On each irreducible spectral sector, the symmetry acts as planar rotation with fixed frequency. On such a sector, the commutant of the symmetry action is commutative \cite{KadisonRingrose,HallQM}.

\subsection{Sectorwise obstruction}

\begin{theorem}[Sectorwise quaternionic obstruction]
On each irreducible symmetry-invariant spectral sector, no pair of anti-commuting symmetry-compatible complex structure operators can exist.
\end{theorem}

\begin{proof}
Let \(\mathcal{S}\) be an irreducible symmetry-invariant spectral sector. Any symmetry-compatible operator must commute with the symmetry generator, hence lie in the commutant of the symmetry action on \(\mathcal{S}\). On \(\mathcal{S}\) this commutant is commutative, so any two symmetry-compatible operators \(J_1,J_2\) must satisfy
\[
J_1 J_2 = J_2 J_1.
\]
This contradicts the quaternionic anti-commutation requirement \(J_1 J_2 = - J_2 J_1\). Therefore no such pair exists on \(\mathcal{S}\).
\end{proof}

\subsection{Multiplicity and non-canonical structures}

When spectral multiplicity exceeds one, the commutant of \(A\) includes additional operators acting within multiplicity subspaces \cite{KadisonRingrose,Dixmier}. Quaternionic structures may exist abstractly in such multiplicity spaces, but they are not determined canonically by one-parameter symmetry alone. Additional independent symmetry generators would be required to canonically select quaternionic structure \cite{AdlerQuaternionicQM,VaradarajanGeometryQM}.

\subsection{Structural consequence}

One-parameter distinguishability symmetry canonically induces complex Hilbert structure (via functional calculus) on \(\mathcal{K}_1\). Quaternionic Hilbert structure is not determined by one-parameter symmetry alone.


\section{Discussion}

\subsection{Completion of symmetry-compatible representational structure}

The prior representational reconstruction established that distinguishability functionals satisfying explicit algebraic consistency conditions admit a canonical real Hilbert-space representation, unique up to isometric isomorphism \cite{WellsMotionFirstQK_v1}. Under the additional assumption of a strongly continuous one-parameter distinguishability symmetry, the present work has shown that this representation admits a canonically determined compatible complex Hilbert structure on the invariant subspace, given the real Hilbert representation and symmetry assumption on the invariant subspace where the symmetry generator acts nontrivially.

This complex structure is determined by the symmetry generator through functional calculus, given the real Hilbert representation and symmetry assumption. On each irreducible symmetry-invariant spectral sector, the symmetry determines a compatible complex structure uniquely up to sign. The sign ambiguity corresponds to complex conjugation and does not affect the isomorphism class of the resulting complex Hilbert space.

Furthermore, this complex structure is representation-intrinsic: the canonical isometric isomorphism relating any two minimal Hilbert representations transports the complex structure consistently. Thus complex scalar multiplication is not an independent representational choice, but follows canonically given the real Hilbert representation together with the one-parameter symmetry assumption.

These results complete the symmetry-compatibility layer of the distinguishability representational framework under the stated symmetry conditions.

\subsection{Spectral structure and sectorwise classification}

The spectral decomposition of the symmetry generator provides a classification of the invariant subspace into symmetry sectors. Each irreducible sector admits a compatible complex Hilbert structure determined canonically by the symmetry generator, given the real Hilbert representation and symmetry assumption.

This decomposition clarifies the internal representational structure of distinguishability theories admitting one-parameter symmetry and shows how complex Hilbert structure follows locally within symmetry-invariant spectral sectors.

In sectors with spectral multiplicity greater than one, additional symmetry-compatible operators may exist within multiplicity subspaces. However, the complex structure defined by functional calculus of the generator remains canonically determined by the generator itself. Additional internal operator structure within multiplicity fibers is not fixed by a single generator alone.

The resulting scalar-field structure is therefore determined by symmetry-compatible representational requirements, rather than introduced independently.

\subsection{Structural limitation of one-parameter symmetry}

The analysis also establishes a structural limitation: one-parameter distinguishability symmetry alone does not canonically determine quaternionic Hilbert structure.

Quaternionic structures may exist abstractly within multiplicity subspaces as algebraic possibilities. However, such structures are not canonically determined or selected by a single symmetry generator. Their canonical determination requires additional independent symmetry generators satisfying noncommutative algebraic relations, specifically anticommuting relations generating the quaternion algebra.

Thus, given the real Hilbert representation and symmetry assumption, complex Hilbert structure represents the maximal scalar-field structure canonically determined by a single strongly continuous one-parameter symmetry.

This identifies both the structural consequences and the structural limitations imposed by one-parameter symmetry.

\subsection{Representational scope}

All results established here are conditional on two structural inputs: the existence of a minimal real Hilbert representation of the distinguishability kernel and the existence of a strongly continuous one-parameter orthogonal symmetry acting on that representation.

Within this framework, the compatible complex Hilbert structure follows canonically on the invariant subspace where the symmetry acts nontrivially, given the real Hilbert representation and symmetry assumption.

No additional dynamical, probabilistic, compositional, or spacetime assumptions are required for this structural result.

\subsection{Role within the broader reconstruction program}

Taken together with the real Hilbert representation established previously \cite{WellsMotionFirstQK_v1}, the present results identify the symmetry-induced scalar-field structure associated with one-parameter distinguishability symmetry.

The representational structure develops in successive conditional layers:

\begin{enumerate}

\item Algebraic distinguishability consistency conditions determine a canonical real Hilbert representation.

\item A strongly continuous one-parameter orthogonal symmetry acting on this representation determines a compatible complex Hilbert structure on irreducible symmetry sectors.

\end{enumerate}

This work does not claim novelty of the analytic construction of complex structure from skew-adjoint generators, which is classical in operator theory \cite{Segal1963,Kostant1970}. Rather, its contribution lies in establishing the canonical and representation-intrinsic status of this structure within the distinguishability representational framework, together with precise uniqueness and structural limitation results.

Further structural layers, including dynamical, compositional, and operational extensions, require additional independent structural assumptions beyond those considered here.


% --------------------------------------------------
% Bibliography
% --------------------------------------------------

\bibliographystyle{plain}
\bibliography{references}

\end{document}
