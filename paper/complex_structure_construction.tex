\section{Construction of the Symmetry-Induced Complex Structure}

We now construct the canonical complex structure operator determined by one-parameter distinguishability symmetry and show that it equips the invariant subspace \(\mathcal{K}_1\) with the structure of a complex Hilbert space.

\subsection{Definition of the complex structure operator}

As established in Section 2, the invariant subspace admits the characterization
\[
\mathcal{K}_1 = \ker(A)^\perp = E((0,\infty))\,\overline{\mathcal{V}}.
\]

Since \(A\) is skew-adjoint, the operator
\[
-A^2
\]
is densely defined, self-adjoint, and positive semidefinite
\cite[Theorem VIII.4]{ReedSimonI}.

By the spectral theorem for self-adjoint operators
\cite[Theorem VII.2]{ReedSimonI}, there exists a projection-valued measure
\(E(\lambda)\) such that
\[
-A^2 = \int_0^\infty \lambda \, dE(\lambda).
\]

On the invariant subspace \(\mathcal{K}_1\), the operator \(-A^2\) has trivial kernel. The functional calculus therefore defines the positive self-adjoint operator
\[
|A| := (-A^2)^{1/2}
=
\int_{(0,\infty)} \lambda^{1/2} \, dE(\lambda),
\]
with dense domain equal to \(\mathcal K_1\)
\cite[Theorem VIII.6]{ReedSimonI}.

The inverse operator
\[
|A|^{-1}
=
(-A^2)^{-1/2}
=
\int_{(0,\infty)} \lambda^{-1/2} \, dE(\lambda)
\]
is defined as a bounded operator on \(\mathcal K_1\) when interpreted via the polar decomposition of the normal operator \(A\), where it appears in the bounded product
\[
J = A|A|^{-1}
\]
defining the partial isometry \(J\)
\cite[Theorem VI.13; Theorem VIII.6]{ReedSimonI}.

Since \(A\) is a closed densely defined operator, it admits a polar decomposition
\[
A = J |A|
\]
where \(J\) is a partial isometry with initial space
\[
\overline{\mathrm{ran}(|A|)} = \ker(A)^\perp = \mathcal{K}_1
\]
and final space
\[
\overline{\mathrm{ran}(A)} = \mathcal{K}_1
\]
\cite[Theorem VI.13]{ReedSimonI}.

Because the initial and final spaces coincide, the partial isometry \(J\) restricts to a unitary operator on \(\mathcal{K}_1\).

Multiplying the polar decomposition on the right by \(|A|^{-1}\) yields
\[
J = A |A|^{-1}
=
A(-A^2)^{-1/2}
\quad \text{on } \mathcal{K}_1.
\]

Since \(|A|^{-1}\) is bounded on \(\mathcal{K}_1\), it follows that \(J\) is bounded on \(\mathcal{K}_1\).

This operator is uniquely determined by the symmetry generator through its polar decomposition, given the real Hilbert representation and symmetry assumption.

\subsection{Algebraic properties of the complex structure operator}

We now establish the fundamental structural properties of \(J\).

\begin{proposition}
The operator \(J\) satisfies
\[
J^2 = -I
\]
on the invariant subspace \(\mathcal{K}_1\).
\end{proposition}

\begin{proof}

Since \(A\) is skew-adjoint, it is a normal operator:
\[
A^*A = AA^*.
\]

Therefore, by the polar decomposition theorem for closed normal operators,
the partial isometry \(J\) in the polar decomposition
\[
A = J|A|
\]
commutes with \(|A|\) and all bounded Borel functions of \(|A|\)
\cite[Theorem VI.13; Theorem VIII.6]{ReedSimonI}.

On the invariant subspace \(\mathcal{K}_1 = \ker(A)^\perp\), the operator
\(|A|\) has trivial kernel and therefore admits a bounded inverse.

Multiplying the polar decomposition on the left by \(|A|^{-1}\) gives
\[
|A|^{-1}A
=
|A|^{-1}J|A|.
\]

Using the commutation of \(J\) and \(|A|\), we obtain
\[
|A|^{-1}J|A|
=
J|A|^{-1}|A|
=
J.
\]

Thus
\[
|A|^{-1}A = J.
\]

Multiplying both sides on the right by \(|A|^{-1}A\) yields
\[
J^2
=
|A|^{-1}A|A|^{-1}A
=
|A|^{-2}A^2.
\]

Since
\[
A^2 = -|A|^2,
\]

it follows that
\[
J^2 = -I
\quad \text{on } \mathcal{K}_1.
\]

\end{proof}

\begin{proposition}
The operator \(J\) is skew-adjoint:
\[
J^* = -J.
\]
\end{proposition}

\begin{proof}

Since \(A\) is skew-adjoint and \(|A|^{-1}\) is self-adjoint,
\[
J^*
=
\left(A|A|^{-1}\right)^*
=
|A|^{-1} A^*
=
-|A|^{-1} A.
\]

Using commutation of \(J\) and \(|A|\),
\[
|A|^{-1}A
=
|A|^{-1}J|A|
=
J.
\]

Therefore,
\[
J^* = -J.
\]

\end{proof}

\begin{proposition}
The operator \(J\) is orthogonal:
\[
\langle J\psi, J\phi \rangle
=
\langle \psi, \phi \rangle
\quad \forall \psi,\phi \in \mathcal{K}_1.
\]
\end{proposition}

\begin{proof}

Using skew-adjointness and \(J^2=-I\),
\[
\langle J\psi, J\phi \rangle
=
\langle \psi, J^*J \phi \rangle
=
\langle \psi, (-J^2)\phi \rangle
=
\langle \psi, \phi \rangle.
\]

\end{proof}

\subsection{Induced complex Hilbert structure}

We now use \(J\) to define complex scalar multiplication on \(\mathcal{K}_1\).

For real scalars \(a,b \in \mathbb{R}\), define
\[
(a+ib)\psi := a\psi + bJ\psi.
\]

Since \(J^2=-I\), this defines a valid complex vector space structure.

Define the complex inner product
\[
\langle \psi, \phi \rangle_{\mathbb{C}}
:=
\langle \psi, \phi \rangle
-
i\langle \psi, J\phi \rangle.
\]

\begin{theorem}[Complex structure compatibility]
Given the real Hilbert representation and strongly continuous symmetry assumption, the invariant subspace \(\mathcal{K}_1\), equipped with this scalar multiplication and inner product, is a complex Hilbert space.
\end{theorem}

\begin{proof}

We verify conjugate symmetry. Using the definition of the complex inner product and skew-adjointness of \(J\), we compute

\begin{align*}
\overline{\langle \phi,\psi\rangle_{\mathbb{C}}}
&=
\overline{
\langle \phi,\psi\rangle
-
i\langle \phi,J\psi\rangle
} \\
&=
\langle \psi,\phi\rangle
+
i\langle \phi,J\psi\rangle \\
&=
\langle \psi,\phi\rangle
-
i\langle \psi,J\phi\rangle \\
&=
\langle \psi,\phi\rangle_{\mathbb{C}},
\end{align*}

where the third line uses skew-adjointness of \(J\), namely
\(
\langle \phi,J\psi\rangle = -\langle J\phi,\psi\rangle = -\langle \psi,J\phi\rangle.
\)

Positive definiteness follows from positivity of the original inner product.

The induced norm satisfies
\[
\|\psi\|_{\mathbb{C}}^2
=
\langle \psi,\psi\rangle_{\mathbb{C}}
=
\|\psi\|^2.
\]

Since \(\mathcal{K}_1\) is closed in the complete real Hilbert space \(\overline{\mathcal{V}}\), it is complete in this norm.

\end{proof}

Thus, given the real Hilbert representation and strongly continuous symmetry assumption, one-parameter distinguishability symmetry determines a canonical complex Hilbert structure on the invariant subspace \(\mathcal{K}_1\).
