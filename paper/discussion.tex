\section{Discussion}

\subsection{Completion of symmetry-compatible representational structure}

The prior representational reconstruction established that distinguishability functionals satisfying explicit algebraic consistency conditions admit a canonical real Hilbert-space representation, unique up to isometric isomorphism \cite{WellsMotionFirstQK_v1}. Under the additional assumption of a strongly continuous one-parameter distinguishability symmetry, the present work has shown that this representation admits a canonically determined compatible complex Hilbert structure on the invariant subspace, given the real Hilbert representation and symmetry assumption on the invariant subspace where the symmetry generator acts nontrivially.

This complex structure is determined by the symmetry generator through functional calculus, given the real Hilbert representation and symmetry assumption. On each irreducible symmetry-invariant spectral sector, the symmetry determines a compatible complex structure uniquely up to sign. The sign ambiguity corresponds to complex conjugation and does not affect the isomorphism class of the resulting complex Hilbert space.

Furthermore, this complex structure is representation-intrinsic: the canonical isometric isomorphism relating any two minimal Hilbert representations transports the complex structure consistently. Thus complex scalar multiplication is not an independent representational choice, but follows canonically given the real Hilbert representation together with the one-parameter symmetry assumption.

These results complete the symmetry-compatibility layer of the distinguishability representational framework under the stated symmetry conditions.

\subsection{Spectral structure and sectorwise classification}

The spectral decomposition of the symmetry generator provides a classification of the invariant subspace into symmetry sectors. Each irreducible sector admits a compatible complex Hilbert structure determined canonically by the symmetry generator, given the real Hilbert representation and symmetry assumption.

This decomposition clarifies the internal representational structure of distinguishability theories admitting one-parameter symmetry and shows how complex Hilbert structure follows locally within symmetry-invariant spectral sectors.

In sectors with spectral multiplicity greater than one, additional symmetry-compatible operators may exist within multiplicity subspaces. However, the complex structure defined by functional calculus of the generator remains canonically determined by the generator itself. Additional internal operator structure within multiplicity fibers is not fixed by a single generator alone.

The resulting scalar-field structure is therefore determined by symmetry-compatible representational requirements, rather than introduced independently.

\subsection{Structural limitation of one-parameter symmetry}

The analysis also establishes a structural limitation: one-parameter distinguishability symmetry alone does not canonically determine quaternionic Hilbert structure.

Quaternionic structures may exist abstractly within multiplicity subspaces as algebraic possibilities. However, such structures are not canonically determined or selected by a single symmetry generator. Their canonical determination requires additional independent symmetry generators satisfying noncommutative algebraic relations, specifically anticommuting relations generating the quaternion algebra.

Thus, given the real Hilbert representation and symmetry assumption, complex Hilbert structure represents the maximal scalar-field structure canonically determined by a single strongly continuous one-parameter symmetry.

This identifies both the structural consequences and the structural limitations imposed by one-parameter symmetry.

\subsection{Representational scope}

All results established here are conditional on two structural inputs: the existence of a minimal real Hilbert representation of the distinguishability kernel and the existence of a strongly continuous one-parameter orthogonal symmetry acting on that representation.

Within this framework, the compatible complex Hilbert structure follows canonically on the invariant subspace where the symmetry acts nontrivially, given the real Hilbert representation and symmetry assumption.

No additional dynamical, probabilistic, compositional, or spacetime assumptions are required for this structural result.

\subsection{Role within the broader reconstruction program}

Taken together with the real Hilbert representation established previously \cite{WellsMotionFirstQK_v1}, the present results identify the symmetry-induced scalar-field structure associated with one-parameter distinguishability symmetry.

The representational structure develops in successive conditional layers:

\begin{enumerate}

\item Algebraic distinguishability consistency conditions determine a canonical real Hilbert representation.

\item A strongly continuous one-parameter orthogonal symmetry acting on this representation determines a compatible complex Hilbert structure on irreducible symmetry sectors.

\end{enumerate}

This work does not claim novelty of the analytic construction of complex structure from skew-adjoint generators, which is classical in operator theory \cite{Segal1963,Kostant1970}. Rather, its contribution lies in establishing the canonical and representation-intrinsic status of this structure within the distinguishability representational framework, together with precise uniqueness and structural limitation results.

Further structural layers, including dynamical, compositional, and operational extensions, require additional independent structural assumptions beyond those considered here.
